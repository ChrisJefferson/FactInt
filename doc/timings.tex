%%%%%%%%%%%%%%%%%%%%%%%%%%%%%%%%%%%%%%%%%%%%%%%%%%%%%%%%%%%%%%%%%%%%%%%%%
%%
%W  timings.tex            GAP4 Package 'FactInt'             Stefan Kohl
%%                                                      
%%%%%%%%%%%%%%%%%%%%%%%%%%%%%%%%%%%%%%%%%%%%%%%%%%%%%%%%%%%%%%%%%%%%%%%%%

\Chapter{How much Time does a Factorization take ?}

%%%%%%%%%%%%%%%%%%%%%%%%%%%%%%%%%%%%%%%%%%%%%%%%%%%%%%%%%%%%%%%%%%%%%%%%%
\Section{The General Factorization Routine}

A few words in advance : in general, it is not possible to give a precise
prediction for the CPU time needed for factoring a given integer.
This time depends heavily on the sizes of the factors of the given number
and some other properties which could not be tested before actually
doing the factorization.
But nevertheless, it is possible to give rough runtime estimates for
numbers with given ``digit partition'' (this means : factors of given
sizes).

By default, after casting out the small and other ``easy'' factors
(which should not take more than at most a few minutes for numbers of
``reasonable'' size) the general factorization routine 
(see ~"IntegerFactorization", ~"FactInt") uses first ECM
(see ~"FactorsECM") for finding factors very roughly up to the third root
of the remaining composite and then the MPQS (see ~"FactorsMPQS")
for doing the ``rest'' of the work (which will often be the most
time-consuming part).

Below I will give some timings for ECM and for the MPQS because
these methods are by far the most important ones in respect to
runtime statistics (the $p \pm 1$ - methods (see ~"FactorsPminus1",
~"FactorsPplus1") are only suitable for finding factors with certain
special properties and CFRAC (see ~"FactorsCFRAC") is just a slower
predecessor of the MPQS). All absolute timings are given for a
Pentium 200 under Windows.

Logfiles for some examples together with their approximate runtimes
can be found on my homepage 
\URL{http://www.cip.mathematik.uni-stuttgart.de/~kohlsn/FactInt/Examples.htm}

%%%%%%%%%%%%%%%%%%%%%%%%%%%%%%%%%%%%%%%%%%%%%%%%%%%%%%%%%%%%%%%%%%%%%%%%%
\Section{Timings for ECM}

The runtime of `FactorsECM' depends mainly on the size of the factors
of the input number.
On average, finding a 12 - digit factor of a 100 - digit number
takes about 1 min 40 s, finding a 15 - digit factor of a 100 - digit
number takes about 10 min and finding an 18 - digit factor of a 100 -
digit number takes about 50 min.
As a general rule of thumb : one digit more increases the run-time
by somewhat less than a factor of two.
These timings are very rough, and they may vary by a factor of 10 or more
(You can compare trying an elliptic curve with throwing a couple of dice,
where a success corresponds to the case where all of them show the same
side - it is possible to be successful with the first trial, but it is
also possible that this takes much longer. In particular, all trials are
independent of one another).
In general, ECM is superior to Pollard's Rho for finding factors with at
least 10 decimal digits and in the same time needed by Pollard's Rho for
finding a 13 - digit factor one can reasonably expect to find a
17 - digit factor when using ECM, for which Pollard's Rho would need 
about 100 times as long as ECM. For larger factors this difference
grows rapidly.
From theoretical calculations it can be stated that finding a
20 - digit factor requires about 500 times as much work as finding a
10 - digit factor, finding a 30 - digit factor requires about 160 times
as much work as finding a 20 - digit factor and finding a 40 - digit
factor requires about 80 times as much work as finding a 30 - digit
factor.

The default parameters are optimized for finding factors with about
15 - 35 digits. This seems to be a sensible choice since this is the
most important range for the application of ECM and for factors
having some digits more or less this should be also not too bad.
(finding a factor with 30 digits or even more requires a lot of
patience - more than I had until now (beginning of May 1999,
a 27 - digit factor is the largest one I have found so far).
ECM usually gives up when the input number <n> has two factors where 
both of them are larger than the third root of <n> (this is certainly
only a rough ``probabilistic'' statement; in general, sometimes 
- but seldom - the remaining composite has 3 factors, 4 factors should
occur (hardly) never). 
Certainly, the user could specify other parameters than 
the default ones - but giving timings for all possible cases here is
obviously impossible. The interested reader should follow the references
given in the bibliography at the end of this manual for getting
information on how much curves with which parameters are usually 
needed for finding factors of a given size. This depends mainly on the
distribution of primes, resp. numbers with prime factors not exceeding a
certain bound.
About the time needed for trying a single curve with given smoothness
bounds for a number of given size (which gives probably the best estimate
for precise runtime comparisons with other implementations) I want to
mention a typical example : one curve with (<Limit1>,<Limit2>) =
(100000,10000000) applied to a 100 - digit integer requires a total of
10 min 20 s, where 6 min 45 s are spent for the first stage
and 3 min 35 s are spent for the second stage.
The time needed for the first stage is approximately linear in <Limit1>
and the time needed for the second stage is somewhat less than linear
in <Limit2>.

%%%%%%%%%%%%%%%%%%%%%%%%%%%%%%%%%%%%%%%%%%%%%%%%%%%%%%%%%%%%%%%%%%%%%%%%%
\Section{Timings for the MPQS}

The runtime of `FactorsMPQS' depends only on the size of the input
number, not on the size of its factors.
Rough timings are : 90 sec for a 40 - digit number, 10 min for a
50 - digit number, 2 h for a 60 - digit number, 20 h for a 70 - digit
number and 100 h for a 75 - digit number (this is the size of the
largest number I factored by the MPQS until now).
These timings are much more precise
than those given for ECM, but they may also vary by a factor of about 2
depending on whether a good factor base could be found without using
a large multiplier or not.
As a general rule of thumb : 10 digits more gives 10 times as much
work. 
For benchmarking purposes, I give the precise timings for some integers
that I have observed : $38! + 1$ (45 digits, good factor base with
multiplier 1) : 2 min 22 s, $40! - 1$ (48 digits, not so good factor
base even with multiplier 7) : 8 min 58 s, cofactor of $1093^{33}+1$
(61 digits, good factor base with multiplier 1) : 1 h 12 min.

%%%%%%%%%%%%%%%%%%%%%%%%%%%%%%%%%%%%%%%%%%%%%%%%%%%%%%%%%%%%%%%%%%%%%%%%%
%%
%E  timings.tex . . . . . . . . . . . . . . . . . . . . . . . . ends here
