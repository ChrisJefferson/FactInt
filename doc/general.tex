%%%%%%%%%%%%%%%%%%%%%%%%%%%%%%%%%%%%%%%%%%%%%%%%%%%%%%%%%%%%%%%%%%%%%%%%%
%%
%W  general.tex             FactInt documentation             Stefan Kohl
%%
%H  @(#)$Id$
%%
%%%%%%%%%%%%%%%%%%%%%%%%%%%%%%%%%%%%%%%%%%%%%%%%%%%%%%%%%%%%%%%%%%%%%%%%%

\Chapter{The General Factorization Routine}

The FactInt package provides a better method for the operation `Factors'
for integer arguments, which supersedes the one included in the {\GAP}
library:

\>Factors( <n> ) M

This method returns a sorted list of the prime factors of <n>.

If it fails to compute the prime factorization of <n>, an error is
signalled.

\index{correctness of the results}
The returned factors pass the built-in probabilistic primality test of
{\GAP} (`IsProbablyPrimeInt', see the {\GAP} Reference Manual).
The same holds for all other factorization routines provided by this
package.

It follows a rough description how FactInt's method for `Factors' works:

First of all it checks whether $n = b^k \pm 1$ for some $b$, $k$ and
looks for factors corresponding to polynomial factors of $x^k \pm 1$.
Provided that $b$ and $k$ are sufficiently small, factors that do not
correspond to polynomial factors are taken from Richard~P.~Brent's
Factor Tables~\cite{Brent04}, which are available at

\URL{http://web.comlab.ox.ac.uk/oucl/work/richard.brent/factors.html.}

Then it uses trial division and a number of cheap methods for special
cases.

After the small and other ``easy'' factors have been found this way,
the method searches for ``medium-sized'' factors using Pollard's Rho
(by the library function `FactorsRho', see the {\GAP} Reference Manual),
Pollard's $p-1$ (see ~"FactorsPminus1"), Williams' $p+1$
(see ~"FactorsPplus1") and the Elliptic Curves Method
(ECM, see ~"FactorsECM") in this order.

If there is still an unfactored part remaining after that,
it is factored using the MPQS (see ~"FactorsMPQS").

The following options are interpreted:

\beginitems
   <TDHints>& A list of additional trial divisors.
   This is useful only if certain primes $p$ are expected to
   divide <n> with probability significantly larger than 
   $\frac{1}{p}$.

   <RhoSteps>& The number of steps for Pollard's Rho.

   <RhoCluster>& The number of steps between two
   gcd computations in Pollard's Rho.

   <Pminus1Limit1> / <Pminus1Limit2>& The first- / second stage
   limit for Pollard's $p-1$ (see ~"FactorsPminus1").

   <Pplus1Residues>& The number of residues to be tried
   by Williams' $p+1$ (see ~"FactorsPplus1").

   <Pplus1Limit1> / <Pplus1Limit2>& The first- / second stage
   limit for Williams' $p+1$ (see ~"FactorsPplus1").

   <ECMCurves>& The number of elliptic curves to be tried by the
   Elliptic Curves Method (ECM) (see ~"FactorsECM").
   Also admissible: a function that takes the number <n> to be
   factored as an argument and returns the desired number
   of curves to be tried.

   <ECMLimit1> / <ECMLimit2>& The initial first- / second stage
   limit for ECM (see ~"FactorsECM").

   <ECMDelta>& The increment per curve for the first stage limit
   in ECM. The second stage limit is adjusted appropriately
   (see ~"FactorsECM").

   <ECMDeterministic>& If true, ECM chooses its curves 
   deterministically, i.e. repeatable (see ~"FactorsECM").

   <FBMethod>& Specifies which of the factor base methods should be
   used to do the ``hard work''. Currently implemented: `"CFRAC"'
   and `"MPQS"' (see ~"FactorsCFRAC", ~"FactorsMPQS"). Default: `"MPQS"'.
\enditems

For the use of the {\GAP} Options Stack, see Section~"ref:Options Stack"
in the {\GAP} Reference Manual.

Setting <RhoSteps>, <Pminus1Limit1>, <Pplus1Residues>, <Pplus1Limit1>,
<ECMCurves> or <ECMLimit1> equal to zero switches the respective method
off. The method chooses defaults for all option values that are not
explicitly set by the user. The option values are also interpreted by
the routines for the particular factorization methods described in the
next chapter.

\beginexample
gap> Factors( Factorial(44) + 1 );
[ 694763, 9245226412016162109253, 413852053257739876455072359 ]
gap> Factors( 2^997 - 1 );
[ 167560816514084819488737767976263150405095191554732902607,
  7993430605360222292860936960123884061988016846627213757686887976005930025638\
602973712891518592878944687759622084106508783413855778177367022158878920741413\
700868182301410439178049533828082651513160945607018874830040978453228378816647\
358334681553 ]
\endexample

The ``working horse'' of this package is

\index{partial factorization}

\>FactInt( <n> ) F

This function computes the prime factorization of the integer <n>.
The result is returned as a list of two lists. The first list
contains the prime factors found, and the second list contains remaining 
unfactored parts of~<n>, if there are any.

`FactInt' interprets all options which are interpreted by the method
for `Factors' described above. In addition, it interprets the option
<FactIntPartial>. If this option is set, the factorization process is
stopped before invoking the (usually time-consuming) MPQS or CFRAC,
if the number of digits of the remaining unfactored part exceeds the
bound passed as option value <MPQSLimit> resp. <CFRACLimit>.

`Factors( n )' is equivalent to
`FactInt( n : FactIntPartial := false )[ 1 ]'.

\beginexample
gap> FactInt( Factorial(120) + 1 : FactIntPartial, ECMCurves := 15, MPQSLimit := 80 );
[ [ 12149, 288790123, 19236668797, 258847257209 ], 
  [ 38291073694445093874467145430046700353728550044101798147114919770210924995\
574407684984595079371248765413916333352628430060046074131773065847748530213870\
5552369545531 ] ]
\endexample

%%%%%%%%%%%%%%%%%%%%%%%%%%%%%%%%%%%%%%%%%%%%%%%%%%%%%%%%%%%%%%%%%%%%%%%%%
%%
%E  general.tex . . . . . . . . . . . . . . . . . . . . . . . . ends here