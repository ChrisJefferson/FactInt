%%%%%%%%%%%%%%%%%%%%%%%%%%%%%%%%%%%%%%%%%%%%%%%%%%%%%%%%%%%%%%%%%%%%%%%%%
%%
%W  general.tex            GAP4 Package 'FactInt'             Stefan Kohl
%%
%%%%%%%%%%%%%%%%%%%%%%%%%%%%%%%%%%%%%%%%%%%%%%%%%%%%%%%%%%%%%%%%%%%%%%%%%

\Chapter{The General Factorization Routines}

%%%%%%%%%%%%%%%%%%%%%%%%%%%%%%%%%%%%%%%%%%%%%%%%%%%%%%%%%%%%%%%%%%%%%%%%%
\Section{If you do not care about the methods used}

\index{factorization without parameters}

\>IntegerFactorization( <n> ) F

`IntegerFactorization' computes the prime factorization of the
integer <n>. The result is returned as a list of the prime factors.
In case of failure an error is signalled.

\index{correctness of the results}
The returned factors are considered to be prime by the built-in
probabilistic primality test of {\GAP} 
(`IsProbablyPrimeInt', see the Reference Manual),
as in all other factorization routines included in this package.

`IntegerFactorization( n )' is equivalent to
`FactInt( n : FactIntPartial:=false )[ 1 ]'
(see ~"FactInt").

`IntegerFactorization' is also installed as a method for `Factors',
with a higher value than the {\GAP} library function `FactorsInt'.

\beginexample
gap> IntegerFactorization( Factorial(24) - 1 );
[ 625793187653, 991459181683 ]
\endexample
\beginexample
gap> Factors( 1459^24 - 1 );
[ 2, 2, 2, 2, 2, 3, 3, 3, 3, 3, 3, 3, 5, 7, 13, 73, 97, 193, 283, 337, 1009, 
  303889, 669433, 1064341, 6722971513, 4531280671081, 313380751265929 ]
\endexample

%%%%%%%%%%%%%%%%%%%%%%%%%%%%%%%%%%%%%%%%%%%%%%%%%%%%%%%%%%%%%%%%%%%%%%%%%
\Section{Taking influence on the methods being used}

\index{factorization with parameters}
\index{partial factorization}

\>FactInt( <n> ) F

`FactInt' computes the prime factorization of the integer <n>.
The result is returned as a list of two lists, where the first one 
contains the prime factors found, and the second one contains
remaining 
\index{unfactored parts}
unfactored parts of <n>,
if there are any (if <FactIntPartial> not set,
this should never happen, because in this case the
function does not return before <n> is factored completely).

The factors returned as prime factors are considered to be prime 
by the build-in probabilistic primality test of {\GAP} 
(`IsProbablyPrimeInt', see the Reference Manual).
This also applies to all other factorization routines 
provided by this package.

`FactInt' recognizes the following options :

\beginitems
   <TDHints>& specifies a list of additional trial divisors
   (when factoring ``random'' numbers, giving a large list
   of trial divisors here is certainly not a sensible approach,
   <TDHints> is useful only if certain primes $p$ are expected to
   divide <n> with probability significantly larger than 
   $\frac{1}{p}$)

   <RhoSteps>&specifies the number of steps for Pollard's Rho,

   <RhoCluster>&specifies the number of steps between two
   Gcd computations in Pollard's Rho

   <Pminus1Limit1>& specifies the first stage limit for Pollard's p-1
   (see ~"FactorsPminus1")

   <Pminus1Limit2>& specifies the second stage limit for Pollard's p-1
   (see ~"FactorsPminus1")

   <Pplus1Residues>& specifies the number of residues
   to be tried by Williams' p+1 (see ~"FactorsPplus1")

   <Pplus1Limit1>& specifies the first stage limit
   for Williams' p+1 (see ~"FactorsPplus1")

   <Pplus1Limit2>& specifies the second stage limit
   for Williams' p+1 (see ~"FactorsPplus1")

   <ECMCurves>& specifies the number of elliptic curves to be 
   tried by the Elliptic Curves Method (ECM) (see ~"FactorsECM"),
   also admissible : a function that takes an argument <n>
   (the number to be factored) and returns the desired number
   of curves to be tried 

   <ECMLimit1>& specifies the initial first stage limit
   for ECM (see ~"FactorsECM")

   <ECMLimit2>& specifies the initial second stage limit
   for ECM (see ~"FactorsECM")

   <ECMDelta>& specifies the increment per curve for the 
   first stage limit in ECM, the second stage limit is adjusted
   appropriately (see ~"FactorsECM")

   <ECMDeterministic>& if true, ECM chooses its curves 
   deterministically, i.e. repeatable (see ~"FactorsECM")

   <FactIntPartial>& if true, the partial factorization obtained
   by applying the factoring methods whose time complexity depends
   mainly on the size of the factors to be found and less
   on the size of <n> (those of the first class mentioned
   in the preface) is returned and the factor base methods
   (MPQS and CFRAC) are not used to complete the factorization
   for numbers that exceed the bound given by <CFRACLimit> 
   resp. <MPQSLimit>; default : false

   <FBMethod>& specifies which of the factor base methods should be
   used to do the ``hard work''; currently implemented : CFRAC and MPQS
   (see ~"FactorsCFRAC", ~"FactorsMPQS")

   <CFRACLimit>& specifies the maximal number of decimal digits of
   an integer to which the Continued Fraction Algorithm (CFRAC) 
   should be applied (only used when <FactIntPartial> = true)
   (see ~"FactorsCFRAC")

   <MPQSLimit>& as above, for the Multiple Polynomial Quadratic 
   Sieve (MPQS) (see ~"FactorsMPQS")
\enditems

Once these option values are set by the user via `PushOptions',
all subsequent factorizations
by `FactInt' and `IntegerFactorization' (see ~"IntegerFactorization")
are done using these settings without the need
to give further arguments (in respect to the use of the
{\GAP} Options Stack, see the Reference Manual).
Setting <RhoSteps>, <Pminus1Limit1>, <Pplus1Residues>, <Pplus1Limit1>,
<ECMCurves> or <ECMLimit1> to zero switches the respective method off.
`FactInt' chooses defaults for all option values that are 
not explicitly set by the user.
The option values are also taken into consideration by the routines for
the particular factorization methods described in the next chapter
when default values have to be chosen. 

If $|n| \< 10^{12}$, `FactInt' just calls the library function
`FactorsInt', which is guaranteed to give the correct result for
numbers in this range. If not, it checks whether $n = b^k \pm 1$ for 
some $b$, $k$ and looks for factors corresponding to
polynomial factors of $x^k \pm 1$ (in order to immediately 
obtain factors that do not correspond to polynomial factors  
it is necessary to give a list of the prime factors of numbers
of this form as <TDHints>).  

As the first general factorization step, `FactInt' does 
\index{trial division}
trial division by the primes below 1000. 
After that, it checks whether the remaining
part of the number to be factored already is a prime.
If not, it does trial divisions by some already known primes, using
the list `Primes2' (see {\GAP} - Manual).
If there is still an unfactored part <m>, then it checks whether <m>
is a non-trivial power of an integer.
Then, the additional list of trial divisors given
as <TDHints> is processed, if present.

After the small and other ``easy'' factors have been casted out 
this way, `FactInt' searches for ``medium - sized'' factors using
Pollard's Rho (by the library function `FactorsRho',
see {\GAP} - Manual),
Pollard's p-1 (see ~"FactorsPminus1"), Williams' p+1 
(see ~"FactorsPplus1") and the Elliptic Curves Method 
(ECM, see ~"FactorsECM") in this order.
After that, if there is still an unfactored part remaining and
<FactIntPartial> = false (or the remaining composites do not
exceed the bound given by <CFRACLimit> resp. <MPQSLimit>), 
one of the factor base methods (CFRAC, see ~"FactorsCFRAC" or 
MPQS, see ~"FactorsMPQS") is used to do the ``hard work'', depending
on the value of <FBMethod>, which could be `"MPQS"' or `"CFRAC"'.

Finally, it is checked whether the product of all factors
is equal to <n> and whether all factors in the first list of the
result pass the {\GAP} pseudoprime test `IsProbablyPrimeInt'
and all factors in the second list are really composites.
These checks are also done by all other factorization routines
provided by this package.  

\beginexample
gap> FactInt( Factorial(39) + 1 : RhoSteps := 16384, Pminus1Limit1 := 100000,
              Pplus1Limit1 := 2000, ECMLimit1 := 5000, ECMCurves := 10,
              FBMethod := "MPQS" );
[ [ 79, 57554485363, 146102648914939, 30705821478100704367 ], [  ] ]
\endexample

%%%%%%%%%%%%%%%%%%%%%%%%%%%%%%%%%%%%%%%%%%%%%%%%%%%%%%%%%%%%%%%%%%%%%%%%%
%%
%E  general.tex . . . . . . . . . . . . . . . . . . . . . . . . ends here
