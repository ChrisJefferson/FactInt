%%%%%%%%%%%%%%%%%%%%%%%%%%%%%%%%%%%%%%%%%%%%%%%%%%%%%%%%%%%%%%%%%%%%%%%%%
%%
%W  preface.tex            GAP4 Package 'FactInt'             Stefan Kohl
%%                                                      
%%%%%%%%%%%%%%%%%%%%%%%%%%%%%%%%%%%%%%%%%%%%%%%%%%%%%%%%%%%%%%%%%%%%%%%%%

\Chapter{Preface}

Factoring large integers is a computationally very difficult problem,
and there is no general factorization algorithm that is capable of
factoring arbitrary integers with more than about 130 - 150 decimal
digits on currently existing computers, but however, there are
methods (not algorithms in the sense that it is guaranteed that
they will give the desired result after a finite number of steps) for 
factoring integers with prime factors being far beyond the range where
trial division is feasible.

One important class of such methods is based on exponentiation in
suitably chosen groups acting on subsets of the $k$ - fold cartesian
product of the set of residue classes (mod $n$), where $n$ denotes the
number to be factored.
Representatives of this class are the Elliptic Curves Method (ECM),
Pollard's p-1 and Williams' p+1.
These methods have the important property that they find smaller
factors usually considerably faster than larger ones
(but with the drawback of being slower for large factors).

The other important class consists of the so-called factor base methods.
These methods compute factorizations of perfect squares (mod $n$) over
an appropriately chosen factor base (a set of small prime numbers, or
of \index{prime ideals}prime ideals in the case of the 
\index{Generalized Number Field Sieve}Generalized Number Field Sieve)
and use them to determine a pair of integers $(x,y)$, such that
$x^2$ and $y^2$ are congruent (mod $n$), but $\pm x$ and $\pm y$ are not.
In this situation, taking Gcd($n$,$x-y$) will yield a non-trivial
factor of $n$.
Representatives of this class are the Continued Fraction Algorithm 
(CFRAC), the Multiple Polynomial Quadratic Sieve (MPQS) and the
already mentioned Generalized Number Field Sieve (GNFS), which has got
the asymptotically lowest complexity of all factoring methods known today
(in respect to \lq difficult' numbers, of course), with the drawback
of being superior to the MPQS only for numbers with more than 100
decimal digits, say, which is probably not within the feasible range
of such a function implemented in {\GAP}.
The first two methods are implemented in this package.

The only method which I am aware of (except of the \lq naive' ones like
trial division and some \lq historical' methods) 
that does not fit into one of the two above - mentioned
classes is \index{Pollard's Rho}Pollard's Rho, 
which is already implemented in the {\GAP} - Library.

In respect to the current state-of-the-art in integer factorization,
see for example the respective web pages of the 
\index{RSA Laboratories}RSA Laboratories
(e.g. http://www.rsa.com/rsalabs/html/status.html).

I would like to thank Bettina Eick and Steve Linton for their support
and many interesting discussions.

I would be grateful for any bug reports, comments or suggestions,

\vfill

\hfill                 Stefan Kohl

%%%%%%%%%%%%%%%%%%%%%%%%%%%%%%%%%%%%%%%%%%%%%%%%%%%%%%%%%%%%%%%%%%%%%%%%%
%%
%E  preface.tex	. . . . . . . . . . . . . . . . . . . . . . . . ends here
