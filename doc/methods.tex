%%%%%%%%%%%%%%%%%%%%%%%%%%%%%%%%%%%%%%%%%%%%%%%%%%%%%%%%%%%%%%%%%%%%%%%%%
%%
%W  methods.tex             FactInt documentation             Stefan Kohl
%%
%H  @(#)$Id$
%%
%%%%%%%%%%%%%%%%%%%%%%%%%%%%%%%%%%%%%%%%%%%%%%%%%%%%%%%%%%%%%%%%%%%%%%%%%

\Chapter{The Routines for Specific Factorization Methods}

Descriptions of the factoring methods used here can be found in
\cite{Bressoud89} and in \cite{Cohen93}.
The last book contains also descriptions of the other methods mentioned
in the preface.

%%%%%%%%%%%%%%%%%%%%%%%%%%%%%%%%%%%%%%%%%%%%%%%%%%%%%%%%%%%%%%%%%%%%%%%%%
\Section{Pollard's p-1}
 
\>FactorsPminus1( <n> [, [ <a>, ] <Limit1> [, <Limit2> ] ] ) F

`FactorsPminus1' tries to factor <n> using Pollard's $p-1$, with <a>
as base for exponentiation (the default is $a=2$), 
<Limit1> as first stage limit and <Limit2> as second stage limit.
`FactorsPminus1' chooses defaults for all arguments not explicitly given.
If `FactorsPminus1' is called with three arguments, these arguments
are regarded as <n>, <Limit1> and <Limit2>.
The result is returned as a list of two lists, where the first one 
contains the prime factors found, and the second one contains
remaining unfactored parts of <n>, if there are any.

Pollard's $p-1$ is based on the fact that exponentiation in the
group of invertible residue classes (mod $n$) is so fast that
it is possible to compute for example $a^{k!}$ for $k$ large enough
(e.g. 100000 or so) in a reasonable amount of time and without
using much memory, and on 
\atindex{Lagrange's Theorem}{@Lagrange's Theorem}
Lagrange's Theorem,
which states that $a^{k!}$ is congruent to 1 (mod $p$) 
(where $p$ is a prime divisor of <n>) if $k!$ is a multiple of $p-1$,
and $(a,p)=1$.
In this situation, if this is satisfied for no other
prime factor of <n>, $p$ can be determined by 
calculating $Gcd(a^{k!}-1,n)$.
A prime divisor $p$ is usually found if the largest prime factor
of $p-1$ is not larger than <Limit2>, and the second-largest one
is not larger than <Limit1>.
(Compare with ~"FactorsPplus1" and ~"FactorsECM".)

\beginexample
gap> FactorsPminus1( Factorial(158) + 1, 100000, 1000000 );
[ [ 2879, 5227, 1452486383317, 9561906969931, 18331561438319, 
      483714299709483760811581110341732950506493218122654853400674921345082310\
906370452295654816571305041217323052879842924826121333143254713674832962773107\
806789945715570386038565256719614524924705165110048148716160964980629081176057\
0095669 ], [  ] ]

# Let's see why this works:

gap> List( last[ 1 ]{[ 3, 4, 5 ]}, p -> IntegerFactorization( p - 1 ) );
[ [ 2, 2, 3, 3, 81937, 492413 ], [ 2, 3, 3, 3, 5, 7, 7, 1481, 488011 ], 
  [ 2, 3001, 7643, 399613 ] ]
\endexample

%%%%%%%%%%%%%%%%%%%%%%%%%%%%%%%%%%%%%%%%%%%%%%%%%%%%%%%%%%%%%%%%%%%%%%%%%
\Section{Williams' p+1}

\>FactorsPplus1( <n> [, [ <Residues>, ] <Limit1> [, <Limit2> ] ] ) F

`FactorsPplus1' tries to factor <n> using a variant of Williams' $p+1$, 
where it tries <Residues> different residues (the probability of a 
given residue to be in principle usable is about 1/2) and 
uses <Limit1> as first stage limit and <Limit2> as second stage
limit. `FactorsPplus1' chooses defaults for all arguments
not explicitly given.
If `FactorsPplus1' is called with three arguments, these arguments
are regarded as <n>, <Limit1> and <Limit2>.
The result is returned as a list of two lists, where the first one 
contains the prime factors found, and the second one contains
remaining unfactored parts of <n>, if there are any.

Williams' $p+1$ is very similar to Pollard's $p-1$ 
(see ~"FactorsPminus1"), the only difference is that the group 
used here has order $p+1$ (instead of $p-1$), and that the group
operation takes more time.
A prime divisor $p$ is usually found if the largest prime factor
of $p+1$ is at most <Limit2> and the second-largest one is not
larger than <Limit1>, and if the algorithm hits a ``usable''
residue (concerning the ``unusable'' residues, it could be stated
that the order of the respective group is $p-1$, such that
the method turns into a slow $p-1$-algorithm in this case).
(Compare also with ~"FactorsECM".)

\beginexample
gap> FactorsPplus1( Factorial(55) - 1, 10, 10000, 100000 );
[ [ 73, 39619, 277914269, 148257413069 ], 
  [ 106543529120049954955085076634537262459718863957 ] ]
gap> List( last[ 1 ], p -> [ Factors( p - 1 ), Factors( p + 1 ) ] );
[ [ [ 2, 2, 2, 3, 3 ], [ 2, 37 ] ], 
  [ [ 2, 3, 3, 31, 71 ], [ 2, 2, 5, 7, 283 ] ], 
  [ [ 2, 2, 2207, 31481 ], [ 2, 3, 5, 9263809 ] ], 
  [ [ 2, 2, 47, 788603261 ], [ 2, 3, 5, 13, 37, 67, 89, 1723 ] ] ]
\endexample

%%%%%%%%%%%%%%%%%%%%%%%%%%%%%%%%%%%%%%%%%%%%%%%%%%%%%%%%%%%%%%%%%%%%%%%%%
\Section{The Elliptic Curves Method (ECM)}

\atindex{Elliptic Curves Method (ECM)}{@Elliptic Curves Method (ECM)}

\>FactorsECM( <n> [, <Curves> [, <Limit1> [, <Limit2> [, <Delta> ] ] ] ] ) F

`FactorsECM' tries to factor <n> using the Elliptic Curves Method (ECM),
where <Curves> specifies the number of curves to be tried,
<Limit1> is the initial
\index{first stage limit}
first stage limit,
<Limit2> is the initial
\index{second stage limit}
second stage limit and
<Delta> is the increment per curve for the first stage limit, where
the second stage limit is adjusted appropriately. 
`FactorsECM' chooses defaults for all arguments not explicitly given.
The option <ECMDeterministic> specifies,
if set, that the choice of the curves to be tried should be 
deterministic, i.e. that repeated calls of `FactorsECM' yield 
the same curves, and hence for the same <n> the result after the same
number of trials (this is of use mainly for testing purposes).
The result is returned as a list of two lists, where the first one 
contains the prime factors found, and the second one contains
remaining unfactored parts of <n>, if there are any.

The Elliptic Curves Method is based on the fact that exponentiation
in the
\index{elliptic curve groups}
elliptic curve groups 
\atindex{E(a,b)/n}{@E(a,b)/n}
$E(a,b)/n$ is fast enough
that it is possible to compute for example $g^{k!}$ for $k$ large enough 
(e.g. 10000 or so) in a reasonable amount of time and without
using much memory, and on Lagrange's Theorem, 
which states that for each
\index{elliptic curve point}
elliptic curve point $g$,
$g^{k!}$ is the identity element of $E(a,b)/p$ (where $p$ is a prime
divisor of <n>) if $k!$ is a multiple of $|E(a,b)/p|$.
In this situation, under reasonable circumstances, $p$ can be
determined by taking an appropriate Gcd.

In practice, the algorithm chooses in some sense ``better''
products $P_k$ of small primes rather than $k!$ as exponents, and,
after reaching the first stage limit with $P_{Limit1}$, it
considers further products $P_{Limit1}q$ for primes $q$ up to
the second stage limit <Limit2> (which is usually set equal to, 
for example, 40 times the first stage limit, or so).
The prime $q$ corresponds to the largest prime factor of the
order of the group under consideration.

A prime divisor $p$ is usually found if the largest prime factor
of the order of one of the examined elliptic curve groups $E(a,b)/p$ 
is at most <Limit2>, and the second-largest one is at most <Limit1>,
so trying a larger number of curves increases the chance of
factoring <n> as well as taking a larger value
for <Limit1> and/or <Limit2> (it turns out to be not optimal either to
take a large number of curves with very small <Limit1> and <Limit2> 
or to grind on with a single curve and very large limits).
(Compare with ~"FactorsPminus1".)

The elements of the group $E(a,b)/n$ are the points $(x,y)$ given by the
solutions of $y^2 = x^3 + ax + b$ in the residue class ring (mod $n$),
and an additional point $\infty$ at infinity, which serves as 
identity element. 
To turn this set into a group, define the product 
(although elliptic curve groups are usually written additively,
I prefer using the multiplicative notation here to retain the analogy
to ~"FactorsPminus1" and ~"FactorsPplus1") of two points
$p_1$ and $p_2$ as follows:
Let $l$ be the line through $p_1$ and $p_2$ if $p_1 \neq p_2$,
otherwise let $l$ be the tangent to the curve $C$ given by the 
above equation in the point $p_1 = p_2$.
$l$ intersects $C$ in a third point, say $p_3$ 
(if $l$ does not intersect the curve in a
third affine point, then set $p_3$ := $\infty$). 
Set $p_1.p_2$ equal to the image of $p_3$ under
the reflection across the $x$-axis.
Define the product of any curve point $p$ and $\infty$ by $p$ itself.
This (more or less obviously, checking associativity requires some
calculation) turns the set of points on the given curve
into an abelian group $E(a,b)/n$.

However, the calculations are done in
\index{projective coordinates}
projective coordinates to have an explicit representation of the 
identity element and to avoid calculating inverses (mod $n$)
for the group operation, which would otherwise require using an 
$O((log n)^3)$-algorithm, while multiplication (mod $n$) is only 
$O((log n)^2)$. The respective equation in this case is given by 
$bY^2Z = X^3 + aX^2Z + XZ^2$ (this form allows more efficient
calculations than the
\atindex{Weierstrass model}{@Weierstrass model}
Weierstrass model 
$Y^2Z = X^3 + aXZ^2 + bZ^3$, which is the projective equivalent to
the affine representation $y^2 = x^3 + ax + b$ mentioned above).
The algorithm only keeps track of two of the three coordinates,
namely $X$ and $Z$.
The choice of curves is done in a way that ensures the order of
the respective group to be divisible by 12. This increases the
chance that it is smooth enough to find a factor of <n>.
The implementation follows the description of R. P. Brent given in
\cite{Brent96}, pp. 5 -- 8 (in terms of this paper,
for the second stage the ``improved standard continuation'' is used).

\beginexample
gap> FactorsECM(2^256+1,100,10000,1000000,100);
[ [ 1238926361552897, 
      93461639715357977769163558199606896584051237541638188580280321 ], [  ] ]
\endexample

%%%%%%%%%%%%%%%%%%%%%%%%%%%%%%%%%%%%%%%%%%%%%%%%%%%%%%%%%%%%%%%%%%%%%%%%%
\Section{The Continued Fraction Algorithm (CFRAC)}

\atindex{Continued Fraction Algorithm (CFRAC)}{@Continued Fraction Algorithm (CFRAC)}

\>FactorsCFRAC( <n> ) F

`FactorsCFRAC' tries to factor <n> using the Continued Fraction
Algorithm (CFRAC), also known as Brillhart-Morrison Algorithm. 
The result is returned as a list of the prime factors of <n>.
In case of failure an error is signalled.

To CFRAC, the same warning applies 
as to the Quadratic Sieve (see ~"FactorsMPQS").

The Continued Fraction Algorithm tries to find integers $x$, $y$,
such that $x^2 \equiv y^2$ (mod $n$), but not $\pm x \equiv \pm y$
(mod $n$). In this situation, taking $Gcd(x - y,n)$ yields a 
non-trivial factor of <n>. For determining such a pair $(x,y)$, 
the algorithm uses the continued fraction expansion of the square root
of <n> (because <n> is usually very large, it is impossible to compute
the whole period of it, but a much smaller number of terms is enough
in this case). If $x_i/y_i$ is a
\index{continued fraction approximation}
continued fraction approximation for the square root of <n>,
then $c_i := x_i^2 - ny_i^2$ has size only about a small constant times
the square root of <n>.
The algorithm tries to find as many $c_i$ as possible which factor 
completely over a chosen
\index{factor base}
factor base 
(a list of small primes) or with only one factor not in the factor base.
The latter ones are usable only if a second $c_i$ with the same
``large factor'' is found
\index{large factors (factor base)}
Then,
\atindex{Gaussian Elimination}{@Gaussian Elimination}
Gaussian Elimination over $GF(2)$ 
is used to determine which of the congruences $x_i^2 \equiv c_i$
(mod $n$) have to be multiplied together to get a congruence
of the desired form  $x^2 \equiv y^2$ (mod $n$), where the involved
matrix $M$ is given by $M_{ij} = 1$, if an odd power of the $j$-th
element of the factor base divides the $i$-th usable factored
value, and $M_{ij} = 0$ otherwise.
For this purpose it is necessary that the number of factored 
$c_i$ is larger than the rank of $M$, which is approximately given by
the size of the factor base.
(Compare with ~"FactorsMPQS".)

\beginexample
gap> FactorsCFRAC( Factorial(34) - 1 );
[ 10398560889846739639, 28391697867333973241 ]
\endexample

%%%%%%%%%%%%%%%%%%%%%%%%%%%%%%%%%%%%%%%%%%%%%%%%%%%%%%%%%%%%%%%%%%%%%%%%%
\Section{The Multiple Polynomial Quadratic Sieve (MPQS)}

\atindex{Multiple Polynomial Quadratic Sieve (MPQS)}{@Multiple Polynomial Quadratic Sieve (MPQS)}

\>FactorsMPQS( <n> ) F

`FactorsMPQS' tries to factor <n> using the Single Large Prime
Variation of the Multiple Polynomial Quadratic Sieve (MPQS).
The result is returned as a list of the prime factors of <n>.
In case of failure an error is signalled.

The intermediate results of a computation could be saved by
interrupting the calculation with [Ctrl][C] and calling `Pause();'
from the break loop. `FactorsMPQS' then pushes all data important
for resuming the computation again as a record <MPQSTmp>
on the options stack. When called again, it will look whether
there is such an appropriate record on the options stack
and get the previously computed factorization data, if so. 
For continuing the factorization process in another 
session, you will have to write this record to a file.
This is done by the function `SaveMPQSTmp( <filename> )',
the file written by this function is readable by the 
standard `Read'-function of {\GAP}.

Caution: The runtime of `FactorsMPQS' depends only on the size of <n>, 
not on the size of its factors, so if a small factor is not found during
the preprocessing which is done before invoking the sieving process,
you will have to wait as long as if <n> would have two prime factors
of roughly equal size.

The Multiple Polynomial Quadratic Sieve
tries to find integers $x$, $y$, such that $x^2 \equiv y^2$ (mod $n$),
but not $\pm x \equiv \pm y$ (mod $n$). In this situation,
taking $Gcd(x - y,n)$ yields a non-trivial factor of <n>.
For determining such a pair $(x,y)$, the algorithm chooses polynomials
$f_a$ of the form $f_a(r) = ar^2 + 2br + c$ with suitably chosen
coefficients $a$, $b$ and $c$, satisfying $b^2 \equiv n$ (mod $a$)
and $c = (b^2 - n)/a$.
The identity $a \times f_a(r) = (ar + b)^2 - n$ yields a congruence
(mod $n$) with a perfect square on one side and $a \times f_a(r)$ on
the other. The algorithm searches for factorizations of polynomial values 
$f_a(r)$ over a chosen factor base (a list of small primes),
either complete or with a single large factor which is not in the
factor base (where the latter ones are only usable if a second $f_a(r)$
with the same ``large factor'' is found) using a sieving technique over
an appropriately chosen
\index{sieving interval}
sieving interval 
(similar to the Sieve of Eratosthenes).
Taking more polynomials and hence shorter sieving intervals gives
the advantage of having smaller $f_a(r)$ to factor over the factor base.
Then, Gaussian Elimination over $GF(2)$ is used to determine the
congruences which have to be multiplied together to get a congruence
of the desired form $x^2 \equiv y^2$ (mod $n$), where the involved 
matrix $M$ is given by $M_{ij} = 1$, if an odd power of the $j$-th 
element of the factor base divides the $i$-th usable factored
value, and $M_{ij} = 0$ otherwise.
For this purpose it is necessary that the number of factored 
$f_a(r)$ is larger than the rank of $M$, which is approximately
given by the size of the factor base.
(Compare with ~"FactorsCFRAC".)

\beginexample
gap> FactorsMPQS( Factorial(38) + 1 );
[ 14029308060317546154181, 37280713718589679646221 ]
\endexample

%%%%%%%%%%%%%%%%%%%%%%%%%%%%%%%%%%%%%%%%%%%%%%%%%%%%%%%%%%%%%%%%%%%%%%%%%
%%
%E  methods.tex . . . . . . . . . . . . . . . . . . . . . . . . ends here
