%%%%%%%%%%%%%%%%%%%%%%%%%%%%%%%%%%%%%%%%%%%%%%%%%%%%%%%%%%%%%%%%%%%%%%%%%
%%
%W  tables.tex             FactInt documentation              Stefan Kohl
%%
%H  @(#)$Id$
%%
%%%%%%%%%%%%%%%%%%%%%%%%%%%%%%%%%%%%%%%%%%%%%%%%%%%%%%%%%%%%%%%%%%%%%%%%%

\Chapter{Getting Factorizations from Precomputed Tables}

How to get factorizations of integers of the form $b^k \pm 1$ quickly
if $b$ is ``small'' and $k$ is rather ``large''?

Many of these numbers have already been factored by various people
using lots of CPU time on fast computers.

Factorizations of integers of this form are frequently needed for
example for computing orders of matrices over finite fields or orders of
elements in huge finite fields. For this reason, this package provides
access to the corresponding tables which have been collected by
Richard~P.~Brent~\cite{Brent04}.

The code for accessing these tables has been written by Frank~L\accent127ubeck.

To make the general factorization routine silently make use of Brent's
tables, you need to download them using the following function:

\>FetchBrentFactors( ) F

This function fetches the current version of the file

\URL{ftp.comlab.ox.ac.uk/pub/Documents/techpapers/Richard.Brent/factors/factors.gz}

and stores the data in various small files in the directory `pkg/factint/tables'.

When `Factors' is called afterwards for an integer of the form
$b^k \pm 1$ for ``small'' $b$, it silently loads the needed part of
the data and uses it for obtaining the requested factorization.

\beginexample
gap> FetchBrentFactors();
gap> Factors(2^997-1);
[ 167560816514084819488737767976263150405095191554732902607,
  7993430605360222292860936960123884061988016846627213757686887976005930025638\
602973712891518592878944687759622084106508783413855778177367022158878920741413\
700868182301410439178049533828082651513160945607018874830040978453228378816647\
358334681553 ]
\endexample

The function `FetchBrentFactors' needs Unix-specific system commands.
Hence it will usually only work under Unix. The reason for not including
the tables in the distribution file is that they would increase its size
by a factor of about~10.

%%%%%%%%%%%%%%%%%%%%%%%%%%%%%%%%%%%%%%%%%%%%%%%%%%%%%%%%%%%%%%%%%%%%%%%%%
%%
%E  tables.tex . . . . . . . . . . . . . . . . . . . . . . . .  ends here